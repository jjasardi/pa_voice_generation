\newpage
\section{Introduction}

\subsection{Initial Situation}

Modern speech synthesis started in the 1960s, it began with large databases of recorded speech that were cut into words and afterwards stitched back together during synthesizing. String of words were then reduced down to the individual diphones, which minimized the size of the databases significantly. In later years there was a statistical approach using a \gls{hmm}, where waveforms are formed based on probability. With the rapid expansion in the field of \gls{dnn} in the past years, voice generation based on \gls{ai} improved immensely as well\cite{enwiki:1189644011}.

More and more successful \gls{ai} voice scams have been reported undeniably because of the increasingly realistic spoofed voice attacks\cite{cbsnews2023voice}.
Impersonation attacks over phone calls can become even easier and identity verification protocols based on voice id will soon be unfeasible. This puts great importance on raising awareness of the capabilities of such attacks and technologies to further improve possible detection or defense strategies and if these render futile, the deprecation of such security systems\cite{vice2023bank}.

\subsection{Objective of this work}
The research question and objectives of this work are based on the official assignment on the subsection \ref{offassignment}.

The primary research question guiding this thesis is:
What is the current state of the art in voice generation, with a particular focus on the quality, ease of use, the potential limitations and risks of open source software implementations? 

To address the research question effectively, the following objectives have been defined:

\begin{itemize}
    \item Review state of the art for voice generation techniques:
    \begin{itemize}
        \item conduct an extensive search of literature on voice generation, focusing on recent developments in the field.
        \item Systematically classify identified voice generation techniques based on their underlying technologies. 
    \end{itemize}
    \item Select and evaluate most recent open source implementations:
    \begin{itemize}
        \item compare the different open source implementations subjectively and objectively through qualitative tests. 
        \item Compare the most recent software against an outdated solution highlighting improvements in recent implementations.
    \end{itemize}
    \item Analyze the potential limitations and risks:
    \begin{itemize}
        \item subjectively assess usability and ease of use, considering installation, intuitiveness of the user interface and accessibility of documentation.
        \item Examine robustness and stability of the implementations, by testing it under potential stress conditions, such as low-quality audio of the voice to be cloned.
        \item Test the synthesised voice on different voice authentication systems to assess its ability to pass authentication checks.
    \end{itemize}
\end{itemize}