\pagenumbering{gobble}

\newpage
\section*{Abstract}

Voice generation through \gls{dnn} has experienced the same advancements with the rise of public attention similar to other breakthroughs achieved by \gls{dnn}s in fields like computer vision. Through voice conversion it is possible to synthesize a human-sounding voice based on short inputs from a targeted speaker. With the increased quality of the end product the differences between real speech and synthesized speech are becoming evanescent.
We study multiple open-source implementations of \gls{dnn} and compare them according to their features, ease of use, subjective quality, visual spectrograms and generation time.
The generated products show a definitive improvements with the latest tools and their success in producing human-sounding voices. However our results also show that the speaker similarity to the target voice aspect is not improving in parallel. Generating a plausible imitation is possible, but not in the time frame that it can be used in an ongoing discussion. This paper concludes that the potential for voice spoofing attacks is higher with the use of \gls{dnn}s and also presents possible defense options against such spoofing attacks.

\paragraph{Keywords:}
Speech Synthesis , text-to-speech, voice conversion, voice cloning, IT security

\clearpage